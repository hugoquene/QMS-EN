% Options for packages loaded elsewhere
\PassOptionsToPackage{unicode}{hyperref}
\PassOptionsToPackage{hyphens}{url}
%
\documentclass[
]{book}
\usepackage{lmodern}
\usepackage{amssymb,amsmath}
\usepackage{ifxetex,ifluatex}
\ifnum 0\ifxetex 1\fi\ifluatex 1\fi=0 % if pdftex
  \usepackage[T1]{fontenc}
  \usepackage[utf8]{inputenc}
  \usepackage{textcomp} % provide euro and other symbols
\else % if luatex or xetex
  \usepackage{unicode-math}
  \defaultfontfeatures{Scale=MatchLowercase}
  \defaultfontfeatures[\rmfamily]{Ligatures=TeX,Scale=1}
\fi
% Use upquote if available, for straight quotes in verbatim environments
\IfFileExists{upquote.sty}{\usepackage{upquote}}{}
\IfFileExists{microtype.sty}{% use microtype if available
  \usepackage[]{microtype}
  \UseMicrotypeSet[protrusion]{basicmath} % disable protrusion for tt fonts
}{}
\makeatletter
\@ifundefined{KOMAClassName}{% if non-KOMA class
  \IfFileExists{parskip.sty}{%
    \usepackage{parskip}
  }{% else
    \setlength{\parindent}{0pt}
    \setlength{\parskip}{6pt plus 2pt minus 1pt}}
}{% if KOMA class
  \KOMAoptions{parskip=half}}
\makeatother
\usepackage{xcolor}
\IfFileExists{xurl.sty}{\usepackage{xurl}}{} % add URL line breaks if available
\IfFileExists{bookmark.sty}{\usepackage{bookmark}}{\usepackage{hyperref}}
\hypersetup{
  pdftitle={Quantitative Methods and Statistics},
  hidelinks,
  pdfcreator={LaTeX via pandoc}}
\urlstyle{same} % disable monospaced font for URLs
\usepackage{longtable,booktabs}
% Correct order of tables after \paragraph or \subparagraph
\usepackage{etoolbox}
\makeatletter
\patchcmd\longtable{\par}{\if@noskipsec\mbox{}\fi\par}{}{}
\makeatother
% Allow footnotes in longtable head/foot
\IfFileExists{footnotehyper.sty}{\usepackage{footnotehyper}}{\usepackage{footnote}}
\makesavenoteenv{longtable}
\usepackage{graphicx}
\makeatletter
\def\maxwidth{\ifdim\Gin@nat@width>\linewidth\linewidth\else\Gin@nat@width\fi}
\def\maxheight{\ifdim\Gin@nat@height>\textheight\textheight\else\Gin@nat@height\fi}
\makeatother
% Scale images if necessary, so that they will not overflow the page
% margins by default, and it is still possible to overwrite the defaults
% using explicit options in \includegraphics[width, height, ...]{}
\setkeys{Gin}{width=\maxwidth,height=\maxheight,keepaspectratio}
% Set default figure placement to htbp
\makeatletter
\def\fps@figure{htbp}
\makeatother
\setlength{\emergencystretch}{3em} % prevent overfull lines
\providecommand{\tightlist}{%
  \setlength{\itemsep}{0pt}\setlength{\parskip}{0pt}}
\setcounter{secnumdepth}{5}
\usepackage{booktabs}
\usepackage{amsthm}
\makeatletter
\def\thm@space@setup{%
  \thm@preskip=8pt plus 2pt minus 4pt
  \thm@postskip=\thm@preskip
}
\makeatother
\usepackage[]{natbib}
\bibliographystyle{apalike}

\title{Quantitative Methods and Statistics}
\author{true}
\date{Version compiled 08 Oct 2020}

\begin{document}
\maketitle

{
\setcounter{tocdepth}{1}
\tableofcontents
}
\hypertarget{preface}{%
\chapter*{Preface}\label{preface}}
\addcontentsline{toc}{chapter}{Preface}

TO BE TRANSLATED

Data spelen een steeds belangrijker rol, ook in de geesteswetenschappen.
De beschikbaarheid van digitale gegevens (o.a. tekst, spraak, video, en gedragsregistraties) leidt tot nieuwe onderzoeksvragen, die vooral met kwantitatieve methoden beantwoord worden.
Dit boek biedt onderzoekers en studenten een overzicht en inleiding van de belangrijkste kwantitatieve methoden en statistische technieken in de geesteswetenschappen. Het boek geeft de lezer een stevig methodologisch fundament voor kwantitatief onderzoek, en biedt een inleiding in de meest gebruikte statistische technieken om gegevens te beschrijven en om hypothesen te toetsen. Daarmee is de lezer ook in staat om kwantitatief onderzoek kritisch te beoordelen.

Dit tekstboek wordt gebruikt als leesstof bij de cursus \emph{Methoden en Statistiek 1} aan de Universiteit Utrecht. Het boek is tevens bruikbaar voor zelfstudie op inleidend niveau, voor iedereen die meer wil weten over kwantitatieve methoden en statistiek.

De hoofdtekst is gevrijwaard van wiskundige afleidingen en formules, die voor geesteswetenschappers immers weinig bruikbaar zijn. De uitleg is vooral conceptueel, en rijk aan voorbeelden van geesteswetenschappelijk onderzoek. Waar nodig worden formules aangeboden in een aparte paragraaf.

Dit boek bevat ook aanwijzingen over hoe de besproken statistische analyses en visualisaties uitgevoerd kunnen worden in twee veelgebruikte programma's, nl. SPSS (versie 22 en later) en R (versie 3.0 en later). Ook deze aanwijzingen staan los van de hoofdtekst, in afzonderlijke paragrafen.

Graag willen we onze mede-docenten danken voor de vele discussies en voorbeelden die op enige wijze verwerkt zijn in dit tekstboek. Onze studenten danken we voor hun nieuwsgierigheid en nauwkeurigheid die geleid heeft tot deze versie van dit tekstboek.

Ook betonen wij grote dank aan
Gerrit Bloothooft,
Margot van den Berg,
Willemijn Heeren,
Caspar van Lissa,
Els Rose,
Tobias Quené,
Kirsten Schutter
en Marijn Struiksma,
voor hun adviezen, data, en/of commentaar bij eerdere versies.

THANK YOU TRANSLATORS: ALEXEI, JOANNA.

Utrecht, October 2020

Hugo Quené, \url{https://www.hugoquene.nl}

Huub van den Bergh

\begin{center}\rule{0.5\linewidth}{0.5pt}\end{center}

\hypertarget{notation}{%
\section*{Notation}\label{notation}}
\addcontentsline{toc}{section}{Notation}

Following international usage we use the point as decimal symbol; hence we write \(\frac{3}{2}=1.5\). Note that the decimal symbol may vary between computers and between software packages on the same computer. Check which decimal symbol is used by (each software package on) your computer.

\hypertarget{license}{%
\section*{License}\label{license}}
\addcontentsline{toc}{section}{License}

This document is licensed under the \emph{GNU GPL 3} license (for details see
\url{https://www.gnu.org/licenses/gpl-3.0.en.html}).

\hypertarget{citation}{%
\section*{Citation}\label{citation}}
\addcontentsline{toc}{section}{Citation}

Please cite this work as follows (in APA style):

Quené, H. \& Van den Bergh, H. (2020). \emph{Quantitative Methods and Statistics}.
Retrieved 8 Oct 2020 from \url{https://hugoquene.github.io/QMS-EN/} .

\hypertarget{technical-details}{%
\section*{Technical details}\label{technical-details}}
\addcontentsline{toc}{section}{Technical details}

The original Dutch version of this text has been written in LaTeX, and was then converted to Rmarkdown, using \texttt{pandoc} \citep{pandoc} and the \texttt{bookdown} \citep{R-bookdown} in \href{https://www.rstudio.com}{Rstudio}. The Dutch version is available at \url{https://hugoquene.github.io/KMS-NL/} .\\
The English translation is based on the Dutch LaTeX version (for Part I) and Rmarkdown version (for Parts II and III).\\
Other versions of this textbook (EPUB, PDF, HTML), the source code (in Rmarkdown) of the text including examples, accompanying datasets, and figures as separate files, are all available at \url{https://github.com/hugoquene/QMS-EN/} .

\hypertarget{about-the-authors}{%
\section*{About the authors}\label{about-the-authors}}
\addcontentsline{toc}{section}{About the authors}

Both authors work at the Faculty of Humanities at Utrecht University, the Netherlands.
HQ is professor in Quantitative Methods of Empirical Research in the Humanities, and he is also founding director of the Centre for Digital Humanities at Utrecht University. HvdB is professor in Didactics and Testing of Language Proficiency, and he is also section chair in Dutch Language and Literature at the Dutch National Board of Tests and Examinations (CvTE).

  \bibliography{book.bib,packages.bib,hhmhto.bib,pandoc.bib}

\end{document}
